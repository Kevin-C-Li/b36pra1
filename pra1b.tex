\documentclass{article}
\usepackage{amsfonts, amssymb, enumitem, amsmath}
\title{Proof of Primes: Testing Primality Using Loops}
\author{Chun Kang Lu, Keethan Jeyabalan, Kevin Li}
\date{2022/09/28}

\begin{document}
	\maketitle
	\flushleft
	\section*{Program}
	Precondition: \(n\in \mathbb{N}, n \geq 2 \) \\ 
	Postcondition: TRUE iff n is prime, FALSE otherwise\\
	is\_prime(n): \\
	\begin{verbatim}
1        i=2; prime = TRUE
2        while(prime and i<n):
3            if(n%i == 0):
4                prime = FALSE
5            else:
6                i = i+1
7        return prime
	\end{verbatim}
	\section*{Loop Invariant}
	\begin{enumerate}[label=(\alph*)]
		\item \(2 \leq i \leq n\)
		\item prime = TRUE \(\Longrightarrow(\forall k \in \mathbb{N}, 2 \leq k < n, n \not\equiv 0 \pmod k)\)
		\item prime = FALSE \(\Longrightarrow (i < n) \land (i\mid n)\)
	\end{enumerate}
	\section*{Proof of Loop Invariant}
	\textbf{$P(i)$: } $\forall n \in \mathbb{N}, n \geq 2, i \in \mathbb{N}$ and conditions (a), (b), and (c) hold.  \\
	\textbf{Basis:} On entering the loop,\\
	\begin{align*}
	i = 2 \land prime = true &&\text{[line 1]}\\
	\end{align*}
	We have that $2 \leq i \leq n$, so (a) holds\\
	prime = TRUE, and no k is in range, so (b) holds\\
	Since (b) holds, (c) is not applicable, 
	$\therefore$ the loop invariant holds.\\
	\textbf{Induction Step:} Consider some arbitrary iteration of the loop, and suppose $P(i)$ holds before the iteration.\\
	We have prime = TRUE and $i<n$. (i)
	\begin{itemize}
		\item For (a), consider [line 3].\\
			\begin{align*}
				n \equiv 0 \pmod{i} &\implies i' = i \leq n &&\text{[by (a)]} \\
				n \not\equiv 0 \pmod{i} &\implies i' = i+1 &&\text{[line 6]}
			\end{align*}
			By (i), $i<n \implies i+1$, thus $i' = i+1 \leq n$ \\As wanted for (a)
		\item For (b), suppose prime$'$ = TRUE. Then, $n \not\equiv 0 \pmod{i}$ by line 3 and $i' = i+1$ by line 6.\\
			Furthermore, we know that $k \in \mathbb{N}, 2 \leq k < i, n \not\equiv 0 \pmod{k}$.\\
			As a result, $\forall k \in \mathbb{N}, 2 \leq k < i' \implies n \not\equiv 0 \pmod{k}$\\
			As wanted for (b)
		\item For (c), suppose prime$'$ = FALSE. We have $n \equiv 0 \pmod{i}$ by line 3 and $i' = i$ since line 6 is not executed.\\
			$0 \equiv n \pmod{i'}$; by (i), $i' = i < n$\\
			As wanted for (c)
	\end{itemize}
	$\therefore$ $P(i) \implies P(i+1)$ and the loop is correct.
	\section*{Partial Correctness}
	Suppose the loop terminates  and consider the values of i, n, and prime on exit.\\
	\begin{enumerate}[leftmargin=.8cm, labelsep=0cm, align=left]
		\item[Case 1: ] prime = TRUE. By (a), $i \leq n$. By exit condition, $i > n$. \\ $\therefore i=n$.\\By (b), $\forall k \in \mathbb{N}, 2 \leq k < n \implies n \not\equiv 0 \pmod{k}$, so n is prime as wanted.
		\item[Case 2: ] prime = FALSE. By (a) and (c), $(2 \leq i < n) \land (n \equiv 0 \pmod{i})$, so i is a factor of n, meaning that n is not prime as wanted.
	\end{enumerate}
	\section*{Termination}
	If prime = FALSE or prime$'$ = FALSE, then the loop terminates.\\
	Suppose prime = TRUE and prime$'$ = TRUE. Associate each iteration with the expression\\ \(e=n-i\) \\By (a), $i \leq n$, so $e = n-i \geq 0 \implies e \in \mathbb{N}$. \\
	Then,
	\begin{align*}
		e' &= n-i' &&\text{[def. of e$'$]}\\
		    &= n-(i+1) &&\text{[line 6]}\\
		    &= n-i-1\\
		    &= e-1 &&\text{[def. of e]}\\
		    &<e\\ 
	\end{align*}
	$\therefore$ The values of $e$ form a strictly decreasing sequence of natural numbers. By the well-ordering principle, the loop will eventually terminate. 
\end{document}
